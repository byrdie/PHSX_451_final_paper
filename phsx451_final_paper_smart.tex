\documentclass[12pt]{article}
 
\usepackage[margin=1in]{geometry} 
\usepackage{amsmath,amsthm,amssymb, mathtools}
\usepackage{multicol}
\usepackage[subnum]{cases}
\usepackage{relsize}
\usepackage[makeroom]{cancel}
\usepackage[english]{babel}
\usepackage{graphicx}
\usepackage{calligra}
\usepackage[normalem]{ulem}
\usepackage{caption}
\usepackage{subcaption}
\usepackage{siunitx}
\usepackage{ mathrsfs }
\usepackage{float}
\usepackage{hyperref}
\usepackage{url}


\DeclareMathAlphabet{\mathcalligra}{T1}{calligra}{m}{n} 
\DeclareFontShape{T1}{calligra}{m}{n}{<->s*[2.2]callig15}{}


% Makes '\sr' make a script r
\newcommand{\sr}{\ensuremath{\mathcalligra{r}}}
 
\newcommand{\N}{\mathbb{N}}
\newcommand{\Z}{\mathbb{Z}}
\newcommand{\ihat}{\boldsymbol{\hat{\textbf{\i}}}}
\newcommand{\jhat}{\boldsymbol{\hat{\textbf{\j}}}}
\newcommand{\khat}{\boldsymbol{\hat{\textbf{k}}}}
\newcommand{\rhat}{\boldsymbol{\hat{\textbf{r}}}}
\newcommand{\srhat}{\boldsymbol{\hat{\textbf{\sr}}}}
\newcommand{\xhat}{\boldsymbol{\hat{\textbf{x}}}}
\newcommand{\yhat}{\boldsymbol{\hat{\textbf{y}}}}
\newcommand{\zhat}{\boldsymbol{\hat{\textbf{z}}}}
\newcommand{\phihat}{\boldsymbol{\hat{\textbf{$\phi$}}}}

\newcommand{\vect}[1]{\boldsymbol{\vec{#1}}}
\newcommand{\fracl}[2]{\mathlarger{\frac{#1}{#2}}}
\newcommand{\dd}{\, \mathrm{d}}
\newcommand{\eo}{\epsilon_\circ}
\newcommand{\mo}{\mu_\circ}
\newcommand{\tder}[2]{\frac{\dd #1}{\dd #2}}
\newcommand{\pder}[2]{\frac{\partial #1}{\partial #2}}
\newcommand{\dtder}[2]{\frac{\dd^2 #1}{\dd #2^2}}
\newcommand{\dpder}[2]{\frac{\partial^2 #1}{\partial #2^2}}
\newcommand{\intas}{ \int_{-\infty}^\infty}

\begin{document}

\title{Engineering and Computational Techniques of Particle Detection Utilized in the Compact Muon Solenoid}
\author{Roy Smart \\ Montana State University \\PHSX 451 Elementary Particle Physics}
\maketitle

\section{Introduction}
On July 4th 2012, CERN hosted a press conference to announce their findings from the Large Hadron Collider (LHC) \cite{website:higgs_press}. While the world waited with baited breath, CERN researchers confirmed they had identified a particle with mass 125 GeV \cite{new_higgs}. This discovery confirmed the existence of the Higgs boson, a particle that has been postulated to exist for at least 60 years \cite{higgs_predict}. 

The LHC utilizes two complementary general-purpose detectors: ATLAS (A Toroidal LHC ApparatuS) and CMS (Compact Muon Solenoid). The two detectors work independently of one another, and in the case of the Higgs particle, provide independent verification of discovery [Source??]. In this paper, we will explore the engineering strategies of the CMS detector that enabled this unique discovery, and the computational techniques used to validate the existence of the Higgs.

\bibliographystyle{plain}
\bibliography{sources}

\end{document}